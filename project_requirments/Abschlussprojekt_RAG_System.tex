\documentclass[12pt]{article}
\usepackage[margin=1in]{geometry}
\usepackage{enumitem}
\usepackage{hyperref}
\usepackage{xcolor}
\usepackage{parskip}
\usepackage{microtype}
\usepackage{lmodern}
\usepackage[ngerman]{babel}
\usepackage{needspace}

\definecolor{linkblue}{RGB}{40, 90, 150}

\hypersetup{
    colorlinks=true,
    linkcolor=linkblue,
    urlcolor=linkblue
}

\title{\textbf{Abschlussprojekt: Entwicklung eines RAG-Systems}}
\author{}
\date{}

\begin{document}

\maketitle

\section*{Überblick}

In dieser Aufgabe entwickeln Sie ein \textbf{Retrieval-Augmented Generation (RAG)}-System, das externe Dokumente verwendet, um Benutzerfragen zu beantworten. Ziel ist es zu zeigen, wie große Sprachmodelle (LLMs) mit aktuellem, domänenspezifischem Wissen verknüpft werden können.

Das verwendete Modell hat einen Wissensstand mit \textbf{Stichtag August 2024}. Ihr System muss daher Informationen aus \textbf{externer Dokumentenrecherche} verwenden, um Fragen zu \textbf{Ereignissen nach diesem Datum} korrekt zu beantworten.

\section*{Projektanforderungen}

\subsection*{Dokumentenauswahl und -verarbeitung}
Wählen Sie ein Dokument über ein Ereignis nach August 2024. Speichern und indexieren Sie es mit \textbf{ChromaDB} und aktivierter Persistenz. Verwenden Sie eine geeignete Textaufteilung, um mindestens \textbf{50 sinnvolle Chunks} zu erzeugen.

\subsection*{Systemarchitektur}
Nutzen Sie das Modell \texttt{gemini-2.0-flash}. Die Pipeline ist mit \textbf{LangChain} oder \textbf{LlamaIndex} umzusetzen. Zur Nachvollziehbarkeit und Beobachtbarkeit verwenden Sie \textbf{LangSmith} oder \textbf{LangFuse}.

Der Einsatz von \textbf{vorgefertigten Agenten ist nicht erlaubt}. Die zentralen Komponenten müssen selbst entwickelt werden.

Das System muss folgende Funktionen unterstützen:
\begin{itemize}[noitemsep]
  \item \textbf{Dialogführung} – mehrstufige Benutzerdialoge.
  \item \textbf{Speicherung von Kontext} – Beibehaltung des Gesprächskontexts über mehrere Eingaben hinweg.
\end{itemize}

\subsection*{Wirksamkeitstest}
Erstellen Sie mindestens \textbf{fünf Fragen}, die nur mit Hilfe des Dokuments korrekt beantwortet werden können. Das Modell soll diese Fragen \textbf{nicht ohne Dokumentretrieval} beantworten können.

Testen Sie Ihr System mit und ohne Retrieval und dokumentieren Sie die Unterschiede. Variieren Sie außerdem die \textbf{System-Prompts} und analysieren Sie deren Auswirkungen.

\subsection*{Codequalität}
Ihr Repository muss gute Entwicklungsstandards einhalten:
\begin{itemize}[noitemsep]
  \item Keine großen Dateien im Git-Verlauf.
  \item Keine geheimen Token in den Commits.
  \item Gut strukturierter und dokumentierter Code.
\end{itemize}

\section*{Abgabehinweise}

\textcolor{linkblue}{\textbf{Abgabefrist:} 11.05 um 23:59 Uhr}

Jede:r Studierende hat einen eigenen Branch im Repository. Öffnen Sie einen \textcolor{linkblue}{\textbf{Pull Request (PR)}} von Ihrem Arbeits-Branch auf den Ihnen zugewiesenen Branch.

Das Repository befindet sich unter folgendem Link: 
\href{https://github.com/hussamalafandi/GenerativeAI-II-Project}{%
\texttt{github.com/hussamalafandi/\\GenerativeAI-II-Project}}

Ihr PR muss Folgendes enthalten:
\begin{itemize}[noitemsep]
  \item Die vollständige Implementierung Ihres RAG-Systems.
  \item Ein Jupyter-Notebook oder Skript mit Demonstration der:
    \begin{itemize}[noitemsep]
      \item Dokumentenindexierung
      \item Dokumentenabfrage (Retrieval)
      \item Fragebeantwortung
      \item Prompt-Variationen
    \end{itemize}
  \item Einen Link zu Ihrem Projekt in \textbf{LangSmith} oder \textbf{LangFuse}.
\end{itemize}

\needspace{5\baselineskip}
\section*{Bonus (Zusatzpunkte)}

Für Zusatzpunkte muss Ihr System beide der folgenden Funktionen enthalten:

\begin{enumerate}[label=\arabic*.]
  \item \textbf{Metadatenfilterung} – Eingrenzung der Dokumentensuche nach Attributen.
  \item \textbf{Multi-Query-Retrieval} – Nutzung mehrerer oder umformulierter Anfragen zur Verbesserung der Antwortqualität.
\end{enumerate}

\needspace{5\baselineskip}
\section*{Abschließender Hinweis}

Ihr System soll nachweisen, dass es nur mit der Hilfe von Dokumentretrieval zu korrekten Antworten kommt. Der Mehrwert Ihres RAG-Systems zeigt sich darin, dass das Sprachmodell allein nicht ausreicht.

\end{document}
