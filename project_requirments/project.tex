\documentclass[12pt]{article}
\usepackage[margin=1in]{geometry}
\usepackage{enumitem}
\usepackage{hyperref}
\usepackage{xcolor}
\usepackage{parskip}
\usepackage{microtype}
\usepackage{lmodern}
\usepackage{needspace}

\definecolor{linkblue}{RGB}{40, 90, 150}

\hypersetup{
    colorlinks=true,
    linkcolor=linkblue,
    urlcolor=linkblue
}

\title{\textbf{RAG System Homework Project}}
\author{}
\date{}

\begin{document}

\maketitle

\section*{Overview}

Your task is to build a \textbf{Retrieval-Augmented Generation (RAG)} system that answers questions using external documents. This demonstrates how large language models (LLMs) can be grounded in up-to-date, domain-specific knowledge.

The chosen model has a knowledge cutoff in \textbf{August 2024}. Therefore, your system must rely on retrieved documents to answer questions about events \textbf{after this date} — not on the model’s internal knowledge.

\section*{Project Requirements}

\subsection*{Document Preparation}
You must choose a document that describes an event that occurred after August 2024. Store and index the document using \textbf{ChromaDB} with persistence enabled. Apply a text-splitting strategy to divide the content into at least \textbf{50 chunks} to enable meaningful retrieval.

\subsection*{System Implementation}
The system must use the \texttt{gemini-2.0-flash} model. You are required to build the RAG pipeline using either \textbf{LangChain} or \textbf{LlamaIndex}, and integrate with either \textbf{LangSmith} or \textbf{LangFuse} for tracing and observability.

\textbf{Pre-built agents are not allowed.} You must implement core components manually.

The system should support:
\begin{itemize}[noitemsep]
  \item \textbf{Dialog flow} – handle multi-turn interactions.
  \item \textbf{Memory} – retain context across turns.
\end{itemize}

\subsection*{Effectiveness Testing}
Design a set of at least \textbf{five questions} that can only be answered with the help of the retrieved document. The model should \textbf{fail or hallucinate} when answering these questions without retrieval, proving the necessity of RAG.

You should also experiment with \textbf{different system prompts} and describe their impact on behavior and results.

\subsection*{Code Quality}
Your repository must follow best practices:
\begin{itemize}[noitemsep]
  \item No large files in Git history.
  \item No secret tokens in commit history.
  \item Code must be documented and reproducible.
\end{itemize}

\section*{Submission Guidelines}

\textcolor{linkblue}{\textbf{Deadline:} 11.05 at 23:59}

Each student has a dedicated branch in the repository. You must open a \textcolor{linkblue}{\textbf{Pull Request (PR)}} from your working branch to your assigned branch before the deadline.

Your PR must include:
\begin{itemize}[noitemsep]
  \item Full implementation of the RAG system.
  \item A Jupyter notebook or script demonstrating:
    \begin{itemize}[noitemsep]
      \item Indexing
      \item Retrieval
      \item Answering
      \item Prompt experimentation
    \end{itemize}
  \item A link to your project in \textbf{LangSmith} or \textbf{LangFuse}.
\end{itemize}

\section*{Bonus Requirements (Extra Credit)}

To receive extra credit, your system must implement both of the following:

\begin{enumerate}[label=\arabic*.]
  \item \textbf{Metadata filtering} – restrict retrieval to relevant document subsets.
  \item \textbf{Multi-query retrieval} – use rephrased or alternative queries to improve retrieval depth.
\end{enumerate}

\section*{Final Note}

Your RAG system must prove its value by answering questions that the model \textbf{cannot answer on its own}. Focus on demonstrating that correct answers emerge only when retrieval is active and relevant.

\end{document}
